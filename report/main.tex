\documentclass[twocolumn,showpacs,%
  nofootinbib,aps,superscriptaddress,%
  eqsecnum,prd,notitlepage,showkeys,10pt]{revtex4-1}
\usepackage{amssymb}
\usepackage{amsmath}
\usepackage{graphicx}
\usepackage{dcolumn}
\usepackage{hyperref}

\begin{document}

\title{Traversability Estimation for a Legged Robot with Dynamic Graph CNN}
\author{Francesco Saverio Zuppichini}
\affiliation{Universita della Svizzera Italiana (USI), Lugano}
\begin{abstract}
  Point clouds are one of the most versatile 3d data representation. Those clouds can be converted into graphs where each point is connected to the neighbors making it possible to
  apply Graph Deep Learning methods. Usually, we are interested in classifying, label each cloud, or segment, label each point. The architecture that archives \emph{state of the art} results in booth tasks is Dynamic Graph CNN \cite{dgcnn}. It utilizes a special Convolution operator applied directly on the graph's edge called \emph{EdgeConv} using booth local and global information. The aim of this project is to first reproduce the results on the famous ModelNet40 \cite{shapenet} dataset with more than ten thousand meshed. Reproducing the paper's results ensure our architecture's correctness.
  Then, we test DGCNN against a vanilla CNN on a dataset composed by heightmaps, images where each pixel is the height value of a terrain region, labeled as traversable or not traversable. Those images were generated letting a legged crocodile-like robot, \emph{Krock}, walking into a simulated environment on synthetic maps and cropping the corresponding terrain patch around each stored trajectory pose. This dataset represents a really interesting playground to explore with architecture is able to extract the most information from the geometry of the terrain and correctly predict their traversability. 

\end{abstract}

\maketitle

\section{Introduction}

Your introduction goes here! Some examples of commonly used commands and features are listed below, to help you get started.

\section{Some \LaTeX{} Examples}
\label{sec:examples}

\subsection{Sections}

Use \texttt{section}s and \texttt{subsection}s to organize your document. \LaTeX{} handles all the formatting and numbering automatically. Use \texttt{ref} and \texttt{label} for cross-references --- this is Section~\ref{sec:examples}, for example.

\subsection{Tables and Figures}

Use \texttt{tabular} for basic tables --- see Table~\ref{tab:widgets}, for example. You can upload a figure (JPEG, PNG or PDF) using the files menu. To include it in your document, use the \texttt{includegraphics} command (see the comment below in the source code).

% Commands to include a figure:
%\begin{figure}
%\includegraphics[width=\textwidth]{your-figure's-file-name}
%\caption{\label{fig:your-figure}Caption goes here.}
%\end{figure}

\begin{table}
\centering
\begin{tabular}{l|r}
Item & Quantity \\\hline
Widgets & 42 \\
Gadgets & 13
\end{tabular}
\caption{\label{tab:widgets}An example table.}
\end{table}

\subsection{Mathematics}

\LaTeX{} is great at typesetting mathematics. Let $X_1, X_2, \ldots, X_n$ be a sequence of independent and identically distributed random variables with $\text{E}[X_i] = \mu$ and $\text{Var}[X_i] = \sigma^2 < \infty$, and let
$$S_n = \frac{X_1 + X_2 + \cdots + X_n}{n}
      = \frac{1}{n}\sum_{i}^{n} X_i$$
denote their mean. Then as $n$ approaches infinity, the random variables $\sqrt{n}(S_n - \mu)$ converge in distribution to a normal $\mathcal{N}(0, \sigma^2)$.

\subsection{Lists}

You can make lists with automatic numbering \dots

\begin{enumerate}
\item Like this,
\item and like this.
\end{enumerate}
\dots or bullet points \dots
\begin{itemize}
\item Like this,
\item and like this.
\end{itemize}

\begin{acknowledgments}

We thank\dots

\end{acknowledgments}

\bibliography{bib}
\end{document}
\documentclass[twocolumn,showpacs,%
  nofootinbib,aps,superscriptaddress,%
  eqsecnum,prd,notitlepage,showkeys,10pt]{revtex4-1}
\usepackage{amssymb}
\usepackage{amsmath}
\usepackage{graphicx}
\usepackage{dcolumn}
\usepackage{hyperref}

\begin{document}

\title{Traversability Estimation for a Legged Robot with Dynamic Graph CNN}
\author{Francesco Saverio Zuppichini}
\affiliation{Universita della Svizzera Italiana (USI), Lugano}
\begin{abstract}
  Point clouds are one of the most versatile 3d data representation. Those clouds can be converted into graphs where each point is connected to the neighbors making it possible to
  apply Graph Deep Learning methods. Usually, we are interested in classifying, label each cloud, or segment, label each point. The architecture that archives \emph{state of the art} results in booth tasks is Dynamic Graph CNN \cite{dgcnn}. It utilizes a special Convolution operator applied directly on the graph's edge called \emph{EdgeConv} using booth local and global information. The aim of this project is to first reproduce the results on the famous ModelNet40 \cite{shapenet} dataset with more than ten thousand meshed. Reproducing the paper's results ensure our architecture's correctness.
  Then, we test DGCNN against a vanilla CNN on a dataset composed by heightmaps, images where each pixel is the height value of a terrain region, labeled as traversable or not traversable. Those images were generated letting a legged crocodile-like robot, \emph{Krock}, walking into a simulated environment on synthetic maps and cropping the corresponding terrain patch around each stored trajectory pose. This dataset represents a really interesting playground to explore with architecture is able to extract the most information from the geometry of the terrain and correctly predict their traversability. 
\end{abstract}
\maketitle
\section{Introduction}
In the last years, Geometric Deep Learning methods have flourished and successfully been applied to point cloud-related tasks such as classification and segmentation. One of these successfully architecture is \emph{Dynamic Graph CNN} proposed by Wang et all \cite{dgcnn}, it utilizes a special convolution layer, called \emph{EdgeConv}, to classify or segment point clouds. With this project, we aim to reproduce the results of the original paper on the classification task and to test its effectiveness on predicting traversability estimation using ground's patches for a legged robot by comparing it to a classic CNN approach.  
\subsection{Model}
In this section we summarized the original paper's architecture, starting by the barebone of DGCNN, the \emph{EdgeConv}.
\subsection{Edge Convolution}
\subsection{Dynamic Graph CNN}
\subsection{Similarties to other architectures}
\section{Experiment}
\subsection{Dataset}
\subsection{Traversability Estimation}
\section{Results}

\section{Conclusion}
\newpage
\bibliography{bib}
\end{document}